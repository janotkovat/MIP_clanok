% Metódy inžinierskej práce

\documentclass[10pt,twoside,slovak,a4paper]{article}

\usepackage[slovak]{babel}
%\usepackage[T1]{fontenc}
\usepackage[IL2]{fontenc} % lepšia sadzba písmena Ľ než v T1
\usepackage[utf8]{inputenc}
\usepackage{graphicx}
\usepackage{url} % príkaz \url na formátovanie URL
\usepackage{hyperref} % odkazy v texte budú aktívne (pri niektorých triedach dokumentov spôsobuje posun textu)

\usepackage{cite}
%\usepackage{times}

\pagestyle{headings}

\title{Vplývajú mobilné technológie na študentov vo výučbe predmetu matematika?\thanks{Semestrálny projekt v predmete Metódy inžinierskej práce, ak. rok 2020/21, vedenie: Ing. Jozef Sitarčík}} % meno a priezvisko vyučujúceho na cvičeniach

\author{Tamara Janotková\\[2pt]
	{\small Slovenská technická univerzita v Bratislave}\\
	{\small Fakulta informatiky a informačných technológií}\\
	{\small \texttt{xjanotkovat@stuba.sk}}
	}

\date{\small 6. november 2020} % upravte



\begin{document}

\maketitle

\begin{abstract}
V dnešnej dobe vysokého technologického pokroku, ktorý sa vyvíja obrovským tempom, by sme sotva hľadali odvetvie, v ktorom mobilné technológie nemajú využitie. Zamerajúc sa na prostredie vzdelávania ide hlavne o integrovanie mobilných zariadení či tabletov do procesu výučby. V širšom ponímaní, naša práca je zameraná na tému vzdelávania pomocou technológií. Presnejšie sa jedná o využitie mobilných technológií vo výučbe predmetu matematika. Už spomínané mobilné zariadenia, dokážu v edukačnom procese nielen nahradiť učebnice, no najmä zefektívniť výučbu, či už pomocou množstva softvérov, ktoré umožnia lepšie pochopenie danej témy, ale i uľahčenie komunikácie medzi študentom a učiteľom alebo študentami navzájom. Z nášho pohľadu je priam na mieste otázka, ktorej sa v článku chceme venovať: ako, a či vôbec dokážu technológie ovplyvniť študijné výsledky študentov a ich osobný postoj k matematike, počas ktorej spomínané technológie využívajú. 
\end{abstract}



\section{Úvod}

Mobilné technológie a mladí ľudia sú od nepamäti považovaní za veľmi blízkych spoločníkov. V obrovskej množine mladých sa vyskytuje len málo jedincov, ktorí s použitím mobilných technológií nemajú skúsenosti. Vo väčšine prípadov mladý človek na používanie mobilných technológií reaguje pozitívne, ba čo viac, považuje ich za akýsi kľúč ku ľahšiemu životu. Naopak, to, čo robí život mladého človeka v jeho očiach ťažkým je mnohokrát štúdium. Idúc do hĺbky, kameňom úrazu býva práve odvetvie matematiky.

Nasledujúce kapitoly nám poskytnú pohľad na kombináciu príjemného s užitočným- kombináciu matematiky a mobilných technológií vo výučbovom procese. V ~\ref{nejaka} kapitole si objasníme čo spomínané mobilné technológie sú, aké konkrétne využitie majú v procese výučby a predstavíme si zopár matematických softvérov a technológii, ktoré sa na matematike využívajú. ~\ref{ucenie} kapitola je venovaná problematike efektívneho učenia.~\ref{dolezita} kapitola sa zameriava na konkrétne používanie mobilných technológií študentami vo výučbe matematiky. Výsledky sú zhrnuté v záverečnej časti.

\section{Mobilné technológie} \label{nejaka}

Slovník cudzích slov definuje pojem technológia nasledovne: „technológia -odbor zaoberajúci sa uplatňovaním prírodovedných, najmä fyzikálnych a chemických poznatkov pri zavádzaní, zdokonaľovaní a využívaní výrobných postupov“\cite{Slovnik}. Prívlastok \emph{mobilný} nesie význam pohyblivý, prenosný. 

V knihe Cognitive Development in Digital Contexts sú mobilné technológie považované za technológie, zahŕňajúce prenosné elektronické zariadenia (ako napr. počítače), ktoré používajú LCD displej na zobrazenie digitálnych obrázkov a sú ovládané dotykom obrazovky pomocou interaktívneho pera alebo pomocou vstupu znakov z digitálnej klávesnice\cite{FIETZER2017167}. Ak tieto dve definície spojíme dohromady, dostávame celkom jasnú predstavu o význame tohto slovného spojenia. 

V našom ponímaní môžeme technológiu chápať ako určitú techniku, respektíve určité mobilné, prenosné vybavenie, ktoré sa uplatňuje na zdokonaľovanie sa. 

\subsection{Mobilné technológie v procese výučby} \label{ina:vyuzitie}


Po objasnení pojmu mobilné technológie by sme chceli predstaviť aspoň niektoré z obrovského množstva možných využití mobilných technológií vo výučbovom procese. 

Ak by sme sa zamerali na výučbový proces vo všeobecnosti, postačujúcim nástrojom k výučbe by sa mohol stať notebook. Dokáže nahradiť nami dlho používané učebnice, či rôzne knihy, ktoré sú v poväčšine dostupné online. Nemusí však ísť len o notebook, pretože na trhu sú ponúkané i varianty ako tablety či čítačky.

Pedagógovia čoraz viac zahŕňajú mobilné technológie do edukačného procesu, najmä kvôli lepšej možnosti pochopenia učiva. Dnes dostupné aplikácie nám ponúkajú rôzne edukatívne videá, vizualizácie procesov, o ktorých sa žiaci dozvedajú počas vyučovania. Čo je dôležité spomenúť sú taktiež aplikácie, ktoré ponúkajú celkový prenos študentských preukazov či žiackych knižiek do virtuálneho sveta.

S výučbovým procesom sú úzko späté aj rôzne testovania a písomky, ktoré sú stále častejšie realizované online formou. Učiteľom je ponúkaná široká škála aplikácií, ktoré umožňujú takéto testovanie. Odhliadnuc od testov, študenti si častokrát merajú svoje vedomosti s ostatnými spolužiakmi počas online kvízov, ktoré si pre nich vyučujúci pripravujú. Pomocou svojich mobilných telefónov sa dokážu pripojiť do virtuálnej miestnosti a odpovedať na dané otázky.  

V neposlednom rade, aj celý výučbový proces dokáže byť realizovaný v online sfére. Príkladom je práve situácia počas pandémie COVID-19, kedy sa základné, stredné a vysoké školstvo na Slovensku prenieslo do online priestorov.

\subsection{Matematické softvéry} \label{ina:softvery}

Okrem už vyššie spomínaných využití mobilných technológií sa zamerajme na také, ktoré sa využívajú pri výučbe predmetu matematika. 

Do množiny mobilných technológií si dovolím zaradiť aj mobilné softvéry ako určitú podmnožinu mobilných technológií. Keďže, aby samotné softvéry boli funkčné, musia byť spustené v jednom zo zariadení.

Jedným z príkladov softvérov prínosných pre predmet matematika je aplikácia GeoGebra. Má využitie hneď vo viacerých okruhoch matematiky počnúc od učiva základnej školy až po univerzity. Dokážeme ju využiť pri štúdiu matematickej analýzy napríklad pri vizualizácii grafov funkcii či geometrie na rysovanie zobrazení.

Ďalšou prínosnou aplikáciou je aplikácia Photomath. Ide o kalkulačku, ktorá dokáže pomocou fotoaparátu zariadenia zoskenovať príklad a následne ho vypočítať. Okrem výsledku nám táto aplikácia ponúkne i postup riešenia so stručným vysvetlením použitých krokov.

Posledným softvérom, ktorý by sme Vám chceli predstaviť je Khan Academy. Táto aplikácia zhromažďuje tisíce edukačných videí z rôznych matematických, i nematematických oblastí a ponúka tak používateľom študijný rast. Čo je veľkým lákadlom, je dostupná zadarmo.

\section{Efektívne učenie} \label{ucenie}

Tu by som rada uviedla spôsoby ako sa študent dokáže efektívnejšie učiť, napríklad na akú formu učiva sa hodí daný spôsob, alebo či sa človeku lepšie učí z farebných poznámok vizualizovaných obrazovke alebo ručne písaných na papier. Samozrejme, pôjde o prepojenie predchádzajúcich kapitol s problematikou, ktorá bude obsiahnutá v tejto kapitole.



\section{Matematika a mobilné technológie} \label{dolezita}

Kapitola č. 4 bude obsahovať konkrétne použitie technológií študentami na matematike. Inšpirovať sa budem pokusom, ktorý realizovali autori článku, ktorý som uviedla ako hlavný zdroj pri spresnení rámcovej témy.

\section{Záver} \label{zaver} 
V tejto kapitole zhrniem všetky zmieňované poznatky a buď potvrdím alebo vyvrátim vplyv mobilných technológií na študentov na predmete matematika.


%\acknowledgement{Ak niekomu chcete poďakovať\ldots}


% týmto sa generuje zoznam literatúry z obsahu súboru literatura.bib podľa toho, na čo sa v článku odkazujete
\bibliography{literatura}
\bibliographystyle{plain} % prípadne alpha, abbrv alebo hociktorý iný
\end{document}
